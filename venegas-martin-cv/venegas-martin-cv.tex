%!TEX TS-program = xelatex
%!TEX encoding = UTF-8 Unicode
% Awesome CV LaTeX Template for CV/Resume
%
% This template has been downloaded from:
% https://github.com/posquit0/Awesome-CV
%
% Author:
% Claud D. Park <posquit0.bj@gmail.com>
% http://www.posquit0.com
%
%
% Adapted to be an Rmarkdown template by Mitchell O'Hara-Wild
% 23 November 2018
%
% Template license:
% CC BY-SA 4.0 (https://creativecommons.org/licenses/by-sa/4.0/)
%
%-------------------------------------------------------------------------------
% CONFIGURATIONS
%-------------------------------------------------------------------------------
% A4 paper size by default, use 'letterpaper' for US letter
\documentclass[11pt,a4paper,]{awesome-cv}

% Configure page margins with geometry
\usepackage{geometry}
\geometry{left=1.4cm, top=.8cm, right=1.4cm, bottom=1.8cm, footskip=.5cm}


% Specify the location of the included fonts
\fontdir[fonts/]

% Color for highlights
% Awesome Colors: awesome-emerald, awesome-skyblue, awesome-red, awesome-pink, awesome-orange
%                 awesome-nephritis, awesome-concrete, awesome-darknight

\colorlet{awesome}{awesome-red}

% Colors for text
% Uncomment if you would like to specify your own color
% \definecolor{darktext}{HTML}{414141}
% \definecolor{text}{HTML}{333333}
% \definecolor{graytext}{HTML}{5D5D5D}
% \definecolor{lighttext}{HTML}{999999}

% Set false if you don't want to highlight section with awesome color
\setbool{acvSectionColorHighlight}{true}

% If you would like to change the social information separator from a pipe (|) to something else
\renewcommand{\acvHeaderSocialSep}{\quad\textbar\quad}

\def\endfirstpage{\newpage}

%-------------------------------------------------------------------------------
%	PERSONAL INFORMATION
%	Comment any of the lines below if they are not required
%-------------------------------------------------------------------------------
% Available options: circle|rectangle,edge/noedge,left/right

\name{Martín}{Venegas M.}

\position{Sociólogo}
\address{Facultad de Ciencias Sociales, Universidad de Chile}

\mobile{+569 84791222}
\email{\href{mailto:martin.venegas@ug.uchile.cl}{\nolinkurl{martin.venegas@ug.uchile.cl}}}
\github{Martin-Venegas-M}
\linkedin{martin-venegas-marquez}

% \gitlab{gitlab-id}
% \stackoverflow{SO-id}{SO-name}
% \skype{skype-id}
% \reddit{reddit-id}


\usepackage{booktabs}

\providecommand{\tightlist}{%
	\setlength{\itemsep}{0pt}\setlength{\parskip}{0pt}}

%------------------------------------------------------------------------------



% Pandoc CSL macros
% definitions for citeproc citations
\NewDocumentCommand\citeproctext{}{}
\NewDocumentCommand\citeproc{mm}{%
  \begingroup\def\citeproctext{#2}\cite{#1}\endgroup}
\makeatletter
 % allow citations to break across lines
 \let\@cite@ofmt\@firstofone
 % avoid brackets around text for \cite:
 \def\@biblabel#1{}
 \def\@cite#1#2{{#1\if@tempswa , #2\fi}}
\makeatother
\newlength{\cslhangindent}
\setlength{\cslhangindent}{1.5em}
\newlength{\csllabelwidth}
\setlength{\csllabelwidth}{3em}
\newenvironment{CSLReferences}[2] % #1 hanging-indent, #2 entry-spacing
 {\begin{list}{}{%
  \setlength{\itemindent}{0pt}
  \setlength{\leftmargin}{0pt}
  \setlength{\parsep}{0pt}
  % turn on hanging indent if param 1 is 1
  \ifodd #1
   \setlength{\leftmargin}{\cslhangindent}
   \setlength{\itemindent}{-1\cslhangindent}
  \fi
  % set entry spacing
  \setlength{\itemsep}{#2\baselineskip}}}
 {\end{list}}

\usepackage{calc}
\newcommand{\CSLBlock}[1]{\hfill\break\parbox[t]{\linewidth}{\strut\ignorespaces#1\strut}}
\newcommand{\CSLLeftMargin}[1]{\parbox[t]{\csllabelwidth}{\strut#1\strut}}
\newcommand{\CSLRightInline}[1]{\parbox[t]{\linewidth - \csllabelwidth}{\strut#1\strut}}
\newcommand{\CSLIndent}[1]{\hspace{\cslhangindent}#1}

\begin{document}

% Print the header with above personal informations
% Give optional argument to change alignment(C: center, L: left, R: right)
\makecvheader

% Print the footer with 3 arguments(<left>, <center>, <right>)
% Leave any of these blank if they are not needed
% 2019-02-14 Chris Umphlett - add flexibility to the document name in footer, rather than have it be static Curriculum Vitae
\makecvfooter
  {julio 2024}
    {Martín Venegas M.~~~·~~~Curriculum Vitae}
  {\thepage}


%-------------------------------------------------------------------------------
%	CV/RESUME CONTENT
%	Each section is imported separately, open each file in turn to modify content
%------------------------------------------------------------------------------



\hypertarget{resumen}{%
\section{Resumen}\label{resumen}}

Sociólogo aficionado a la estadística, los métodos cuantitativos y la
ciencia social abierta. He tenido la oportunidad de aprender sobre estos
tópicos a través de distintos cursos, trabajos y formación autodidacta
durante mi trayectoria universitaria y laboral, permitiéndome formar una
base sólida en lo que respecta al análisis de datos, uso de software y
herramientas para la accesibilidad y reproducibilidad en la
investigación. Abierto a toda instancia de aprendizaje que me permita
poder generar un impacto positivo en la sociedad y en las personas.

\hypertarget{estudios}{%
\section{Estudios}\label{estudios}}

\begin{cventries}
    \cventry{Diplomado en Estadística, mención Métodos Estadísticos}{Pontificia Universidad Católica de Chile}{Santiago, Chile}{2023-2024}{}\vspace{-4.0mm}
    \cventry{Sociólogo}{Universidad de Chile}{Santiago, Chile}{2017-2022}{}\vspace{-4.0mm}
    \cventry{Licenciatura en Sociología}{Universidad de Chile}{Santiago, Chile}{2017-2020}{}\vspace{-4.0mm}
\end{cventries}

\hypertarget{experiencia-laboral}{%
\section{Experiencia laboral}\label{experiencia-laboral}}

\begin{cventries}
    \cventry{Analista Socioeconómico}{Instituto Nacional de Estadísticas}{Santiago, Chile}{2023-presente}{\begin{cvitems}
\item Construcción de instrumentos, diseño y ejecución del procesamiento de datos para la II Encuesta Nacional Sobre Uso del Tiempo (ENUT) del Subdepartamento de Estadísticas Estructurales del Trabajo. Coordinador Técnico del proyecto Mg. Agustín Arce.
\end{cvitems}}
    \cventry{Analista Socioeconómico}{Instituto Nacional de Estadísticas}{Santiago, Chile}{2022}{\begin{cvitems}
\item Diseño de instrumentos, procesamiento y análisis de datos para la VII Encuesta de Microemprendimiento (EME) del Subdepartamento de Estadísticas Estructurales del Trabajo. Coordinador Técnico del proyecto Mg. Sebastián Palacios.
\end{cvitems}}
    \cventry{Practicante}{Instituto Nacional de Estadísticas}{Santiago, Chile}{2022}{\begin{cvitems}
\item Apoyo en levantamiento y análisis de datos al Equipo Técnico de la Encuesta de Microemprendimiento (EME) en la piloto de su VII versión. Coordinador Técnico del proyecto Mg. Sebastián Palacios.
\end{cvitems}}
    \cventry{Asitente de investigación}{Centro de Estudios de Conflicto y Cohesión Social}{Santiago, Chile}{2021-2022}{\begin{cvitems}
\item Asistencia de investigación para investigador post-doctoral Phd. Juan Diego García Castro. Principales actividades: revisión de literatura, manejo de bases de datos y escritura de artículos científicos.
\end{cvitems}}
    \cventry{Pasante de investigación}{Laboratorio de Ciencia Social Abierta (LISA) - Centro de Estudios de Conflicto y Cohesión Social (COES)}{Santiago, Chile}{2021}{\begin{cvitems}
\item Apoyo en sistematización de información y escritura para documento de trabajo sobre transparencia y reproducibilidad en las ciencias sociales. Investigador principal Phd. Juan Carlos Castillo.
\end{cvitems}}
    \cventry{Ayudante de investigación}{Fondecyt Regular 1181239}{Santiago, Chile}{2020-2022}{\begin{cvitems}
\item Apoyo técnico en elaboración de artículos: procesamiento y análisis de datos, revisión de literatura y escritura académica. Investigador principal Phd. Cristian Cox.
\end{cvitems}}
    \cventry{Pasante de investigación}{Fundación Todo Mejora}{Santiago, Chile}{2020}{\begin{cvitems}
\item Desarrollo y análisis de datos para Informe Bianual Programa Hora Segura 2019-2020 y otras actividades. Coordinadora Fernanda Barriga.
\end{cvitems}}
    \cventry{Socio-Investigador}{Nucleo de Sociología Contingente (NUDESOC)}{Santiago, Chile}{2019-actual}{\begin{cvitems}
\item Desarrollo de investigaciones en temáticas contingentes, especialmente acción colectiva, movimientos sociales y política. Presidenta del núcleo Camila Diaz.
\end{cvitems}}
    \cventry{Tutor de Apoyo al Aprendizaje - LEA}{Centro IDEA, Facultad de Ciencias Sociales, Universidad de Chile}{Santiago, Chile}{2019-2021}{\begin{cvitems}
\item Acompañamiento y orientación en estrategias de estudio, escritura y lectura académica y contenidos de cátedra para estudiantes de Sociología (focalizado en ciclo básico). Encargada del programa Mg. Carla Gutierrez.
\end{cvitems}}
    \cventry{Encuestador}{Semsum Consultoría}{Santiago, Chile}{2018}{\begin{cvitems}
\item Aplicación de cuestionario para levantamiento de información respecto a opiniones en torno a Municipalidad de Lampa. Encargado Julián Goren.
\end{cvitems}}
    \cventry{Transcriptor}{Conexium Consultoría y Cía. Ltda.}{Santiago, Chile}{2017}{\begin{cvitems}
\item Encargado de transcribir entrevistas en profundidad y grupos focales para proyecto de investigación. Encargado de Gestión de Proyectos Luis Silva Gonzalez.
\end{cvitems}}
\end{cventries}

\hypertarget{experiencia-acaduxe9mica}{%
\section{Experiencia académica}\label{experiencia-acaduxe9mica}}

\begin{cventries}
    \cventry{Ayudante de la cátedra Técnicas Estadísticas Avanzadas}{Facultad de Ciencias Sociales, Universidad Diego Portales}{Santiago, Chile}{Segundo Semestre 2023}{\begin{cvitems}
\item Apoyar a estudiantes de primer año del Magister en Métodos para la Investigación Social en la implementación de técnicas basadas en modelos de ecuaciones estructurales. Profesora responsable Phd. Monica Gerber.
\end{cvitems}}
    \cventry{Ayudante de la cátedra Diseño de Técnicas Cuantitativas}{Facultad de Ciencias Sociales, Universidad Diego Portales}{Santiago, Chile}{Primer Semestre 2022}{\begin{cvitems}
\item Apoyo y asesorías a estudiantes de tercer y cuarto año de la carrera de Sociología respecto a diseños cuantitativos de investigación social, incluyendo diseño de herramientas, levantamiento de datos y redacción de informes. Profesora responsable Phd. Macarena Orchard
\end{cvitems}}
    \cventry{Ayudante de la cátedra Investigación Evaluativa}{Facultad de Ciencias Sociales, Universidad de Chile}{Santiago,Chile}{Primer Semestre 2021}{\begin{cvitems}
\item Apoyo y asesorías a estudiantes de cuarto año de la carrera de Sociología respecto elaboración de diseños de evaluación social. Profesora responsable Phd. Andrea Peroni.
\end{cvitems}}
    \cventry{Ayudante de la cátedra Sociología de las Políticas Públicas}{Facultad de Ciencias Sociales, Universidad de Chile}{Santiago, Chile}{Segundo Semestre 2020}{\begin{cvitems}
\item Apoyo y asesorías a estudiantes de tercer año de la carrera de Sociología respecto a diseños de intervención social, elaboración de diagnosticos y uso de Marco Lógico. Profesor responsable Phd. Fernando Cámpos.
\end{cvitems}}
    \cventry{Ayudante de la cátedra Estadística Multivariada}{Facultad de Ciencias Sociales, Universidad de Chile}{Santiago, Chile}{Primer Semestre 2020}{\begin{cvitems}
\item Apoyo y asesorías a estudiantes de tercer año de la carrera de Sociología respecto a técnicas de análisis multivariado y desarrollo de investigación académica cuantitativa. Profesor responsable Phd. Juan Carlos Castillo.
\end{cvitems}}
    \cventry{Ayudante de la cátedra Estrategias de Investigación Cuantitativa}{Facultad de Ciencias Sociales, Universidad de Chile}{Santiago, Chile}{Segundo Semestre 2019}{\begin{cvitems}
\item Apoyo y asesoría a estudiantes de segundo año de la carrera de Sociología respecto a métodología cuantitativa, especialmente en lo referido a construcción de herramientas, levantamiento de datos y analisis de datos. Profesor responsable Phd. Juan Carlos Castillo.
\end{cvitems}}
\end{cventries}

\hypertarget{aptitudes}{%
\section{Aptitudes}\label{aptitudes}}

\begin{itemize}
\tightlist
\item
  Buena organización de tareas y planificación del tiempo.
\item
  Habilidades en enseñanza-aprendizaje horizontal y dinámicas.
\item
  Habilidades en comunicación oral.
\item
  Habilidades en investigación y elaboración de informes.
\item
  Habilidades en procesamiento y análisis de bases de datos.
\item
  Habilidades en elaboración de instrumentos de levantamiento de datos.
\end{itemize}

\hypertarget{conocimientos-tuxe9cnicos}{%
\section{Conocimientos técnicos}\label{conocimientos-tuxe9cnicos}}

\begin{itemize}
\tightlist
\item
  Manejo avanzado de software estadístico R.
\item
  Manejo avanzado de lenguaje Markdown y versionamiento
  Git/Github/Gitlab.
\item
  Manejo avanzado de gestores bibliográficos Mendeley y Zotero.
\item
  Manejo intermedio de Microsoft Office
\item
  Manejo básico de Survey Solutions.
\item
  Manejo básico de Python.
\item
  Manejo básico de Atlas.Ti
\item
  Manejo de inglés: lectura y escucha fluida, habla intermedia.
\item
  Manejo básico de Portugués.
\end{itemize}

\hypertarget{cursos}{%
\section{Cursos}\label{cursos}}

\begin{cventries}
    \cventry{Ciencias Sociales Computacionales: Análisis de Redes Sociales y Procesamiento de Lenguaje Natural}{Facultad de Ciencias Sociales, Universidad de Chile}{Santiago, Chile}{2020}{}\vspace{-4.0mm}
    \cventry{Ciencia Social Abierta: Herramientas para la Reproducibilidad, Colaboración y Comunicación de la Investigación Social}{Facultad de Ciencias Sociales, Universidad de Chile}{Santiago, Chile}{2020}{}\vspace{-4.0mm}
    \cventry{Data Science: R Basics}{HarvardX}{EdX}{2020}{}\vspace{-4.0mm}
    \cventry{Python for Statistical Analysis}{SuperDataScience Team}{Udemy}{2020}{}\vspace{-4.0mm}
    \cventry{Análisis Multinivel}{Facultad de Ciencias Sociales, Universidad de Chile}{Santiago, Chile}{2019}{}\vspace{-4.0mm}
    \cventry{Introducción a la Ciencia de los Datos}{Facultad de Ciencias Sociales, Universidad de Chile}{Santiago, Chile}{2019}{}\vspace{-4.0mm}
\end{cventries}

\hypertarget{colaboraciuxf3n-y-autoruxedas}{%
\section{Colaboración y autorías}\label{colaboraciuxf3n-y-autoruxedas}}

\hypertarget{refs-b585e8bb1c095d4b76a9060900dcda7a}{}
\begin{CSLReferences}{0}{0}
\leavevmode\vadjust pre{\hypertarget{ref-garcia2024trust}{}}%
\CSLLeftMargin{1. }%
\CSLRightInline{Garcı́a-Castro, J. D., Venegas Márquez, M., \&
Pérez-Ahumada, P. (2024). Trust in unions drives egalitarianism:
Longitudinal evidence. \emph{Journal of Community \& Applied Social
Psychology}, \emph{34}(4), e2831.}

\leavevmode\vadjust pre{\hypertarget{ref-garcia2023evaluacion}{}}%
\CSLLeftMargin{2. }%
\CSLRightInline{Garcı́a-Castro, J. D., Venegas Márquez, M., Ramı́rez
Cardoza, L., \& Robles Rivera, F. (2023). Evaluación de (in) justicia
distributiva en jóvenes de centroamérica. \emph{Andamios},
\emph{20}(52), 363--386.}

\leavevmode\vadjust pre{\hypertarget{ref-castillo_Perception_2021}{}}%
\CSLLeftMargin{3. }%
\CSLRightInline{Castillo, J., García-Castro, J., \& Venegas, M. (2021).
Perception of economic inequality: Concepts, associated factors and
prospects of a burgeoning research agenda (Percepción de desigualdad
económica: Conceptos, factores asociados y proyecciones de una agenda
creciente de investigación). \emph{International Journal of Social
Psychology}. \url{https://doi.org/10.1080/02134748.2021.2009275}}

\leavevmode\vadjust pre{\hypertarget{ref-garcia-sanchez_Ideological_2022}{}}%
\CSLLeftMargin{4. }%
\CSLRightInline{García-Sanchez, E., García-Castro, J. D., Venegas, M.,
\& Castillo, J. C. (2022). The Ideological Underpinnings of Distributive
Unfairness Evaluations: Evidence from Latin America between 1997 and
2020. In H. Cakal, V. Smith, \& D. Sirlopu (Eds.), \emph{Latin America:
Social Psychological Perspectives}.
\CSLBlock{Work in Progress}}

\leavevmode\vadjust pre{\hypertarget{ref-nudesocInformeResultadosOficial2020}{}}%
\CSLLeftMargin{5. }%
\CSLRightInline{Nudesoc. (2020). \emph{Informe de resultados oficial
Encuesta Zona Cero} {[}Preprint{]}. Open Science Framework.
\url{https://doi.org/10.31219/osf.io/76mdz}}

\leavevmode\vadjust pre{\hypertarget{ref-nudesocReporteMetodologicoEncuesta2020}{}}%
\CSLLeftMargin{6. }%
\CSLRightInline{Nudesoc. (2020). \emph{Reporte metodologico Encuesta
Zona Cero} {[}Preprint{]}. Open Science Framework.
\url{https://doi.org/10.31219/osf.io/45j9m}}

\leavevmode\vadjust pre{\hypertarget{ref-roesslerInformeAccesoVivienda2020}{}}%
\CSLLeftMargin{7. }%
\CSLRightInline{Roessler, P., Ramaciotti, J. P., Bravo, S., Faiguenbaum,
M., Pereira, I. O., Leyton, V., Laffert, A., Munoz, B., Venegas, M.,
Campos, F., Pedemonte, N. R., Lagos, T., \& Vargas, F. (2020).
\emph{Informe 3: Acceso a la vivienda y condiciones de habitabilidad de
la poblacion migrante en Chile. Servicio Jesuita a Migrantes -
TECHO-Chile - Facultad de Sociologia U. de Chile - Centro de Etica y
Reflexion Social Fernando Vives SJ, U. Alberto Hurtado}. Unpublished.}

\leavevmode\vadjust pre{\hypertarget{ref-todomejoraInformeProgramaHora2020}{}}%
\CSLLeftMargin{8. }%
\CSLRightInline{Todo Mejora. (2020). \emph{Informe Programa Hora Segura
2019-2020}.}

\end{CSLReferences}

\hypertarget{experiencias-de-formaciuxf3n-personal}{%
\section{Experiencias de formación
personal}\label{experiencias-de-formaciuxf3n-personal}}

\begin{cventries}
    \cventry{Voluntariado regular}{Santuario Clafira}{Limache, Chile}{2019-2020}{}\vspace{-4.0mm}
    \cventry{Centro de Estudiantes de Facultad}{Facultad de Ciencias Sociales, Universidad de Chile}{Santiago, Chile}{2018}{}\vspace{-4.0mm}
    \cventry{Coordinador Núcleo Organizador Escuela Mechona}{Facultad de Ciencias Sociales, Universidad de Chile}{Santiago, Chile}{2018}{}\vspace{-4.0mm}
\end{cventries}



\end{document}
